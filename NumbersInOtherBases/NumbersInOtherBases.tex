\documentclass[handout]{ximera}
%% handout
%% space
%% newpage
%% numbers
%% nooutcomes
\title{Numbers in other bases - practice}
\begin{document}
\begin{abstract} {Sharpen your calculation skills in other bases}
\end{abstract}
\maketitle

For reference, our convention for base-$26$ is to use the letters $A-Z$ as the digits $0-25$:

\begin{center}
\begin{small}
\begin{tabular}{|c|c|c|c|c|c|c|c|c|c|c|c|c|}
\hline
 A & B & C & D & E & F & G & H & I& J & K & L &M  \\
 \hline
0 & 1& 2 & 3 & 4 & 5 & 6 & 7 & 8 & 9 & 10 & 11 & 12 \\
\hline
\hline
N&O&P&Q&R&S&T&U&V&W&X&Y&Z\\
\hline
13 & 14 & 15 & 16 & 17 & 18 & 19 & 20 & 21 & 22 & 23 & 24 & 25\\
\hline
\end{tabular}
\end{small}
\end{center}

Instructions: Calculate the sum, difference, product, or quotient and remainder.  Your answer should be in the starting base unless otherwise specified.

\begin{question}
\[(143)_7 + (53)_7\]

\begin{hint} Just like with base-10 addition, you'll sometimes have to carry a digit.  Here, the digits are $0, 1, 2, 3, 4, 5, 6$.

\begin{prompt} What is $4+5$ in base-$10$? \answer{9}\end{prompt}

\begin{prompt} What is its base-$7$ representation? \answer{(12)_7} \end{prompt}

\begin{prompt} What digit do you have to carry to the next step? \answer{1} 
\end{prompt}
\end{hint}

\answer{(226)_7}
\end{question}

\begin{question}
\[(4671)_8 + (327)_8\]

\answer{(5220)_8}
\end{question}

\begin{question}
\[(APPLE)_{26}+(WINDOWS)_{26}\]

\answer{(WINTEHW)_{26}}
\end{question}

\begin{question} \[(451)_9 - (332)_9\]

\answer{(118)_9}

\end{question}

\end{document}
